%%%%%%%%%%%%%%%%%%%%%%%%%%%%%%%%%%%%%%%%%
% Important note:
% This template requires the resume.cls file to be in the same directory as the
% .tex file. The resume.cls file provides the resume style used for structuring the
% document.
%
%%%%%%%%%%%%%%%%%%%%%%%%%%%%%%%%%%%%%%%%%

%----------------------------------------------------------------------------------------
%	PACKAGES AND OTHER DOCUMENT CONFIGURATIONS
%----------------------------------------------------------------------------------------

\documentclass{resume} % Use the custom resume.cls style
\usepackage{hyperref}
\usepackage[left=0.75in,top=0.6in,right=0.75in,bottom=0.6in]{geometry} % Document margins
\newcommand{\tab}[1]{\hspace{.2667\textwidth}\rlap{#1}}
\newcommand{\itab}[1]{\hspace{0em}\rlap{#1}}
\name{Thai Thien} % Your name
\address{Ho Chi Minh city, Viet Nam} % Your address
%\address{123 Pleasant Lane \\ City, State 12345} % Your secondary addess (optional)
\address{(+84) 082 449 7847\\ thien.thai.work@gmail.com} % Your phone number and email

\begin{document}

%----------------------------------------------------------------------------------------
%	EDUCATION SECTION
%----------------------------------------------------------------------------------------

% \begin{rSection}{Education}
% %--copy and paste this region  if you need more--
% {\bf VNUHCM - University of Science} \hfill {\em 2013 - 2017 } 
% \\ Bachelor of Computer Science % \hfill { GPA: your gpa }
% %--copy and paste this region  if you need more--

% %--copy and paste this region  if you need more--
% {\bf VNUHCM - University of Science} \hfill {\em 2019 - 2021 } 
% \\ Master of Science in Information System % \hfill { GPA: your gpa }

% %--copy and paste this region  if you need more--

% \end{rSection}


\begin{rSection}{Skills}
{\bf Programming Languages and Frameworks }
\\ - Comfortable with Java, Python, shell script
\\ - Data Science: Data transformation and modeling with Numpy, Pandas, Scikit-learn, lightgbm ,Autogluon
\\ - Big Data: Experienced in building data pipeline with Apache Hadoop, Apache Spark and Luigi (Spotify).
\\ - Deep learning: Worked with Pytorch. I did some research on NLP (Bachelor degree research) and Computer Vision (master degree research). Read my research papers for more details.
\\ - Backend: Build microservice with SpringBoot, Java Servlet, Python flask, Python fastapi, RPC (Apache Thrift, gRPC), REST API. 
\\ - Database: Worked with MySQL, MongoDB, Neo4j. 
\\ - Devops: Worked with git, docker/docker-compose, and Gitlab CI. 
\end{rSection}



%----------------------------------------------------------------------------------------
%	EXPERIENCE SECTION
%----------------------------------------------------------------------------------------
\begin{rSection}{Work Experience}
%--copy and paste this region  if you need more--

{\bf \underline{Android Developer at Piksal JSC}} \hfill {\em 2015 - 2016}\\
- Develop variable applications and games on Android. Work with native Java Android, React Native, and Unity Game Engine. 

{\bf \underline{Backend Developer at ShareCarForAds.com}} \hfill {\em 2016 - 2017}\\
- I worked as backend developer. I built backend service to help with data ingestion from IoT devices. 

{\bf \underline{Data Scientist at VNG}} \hfill {\em 2017 - now}
% \\ - Work on variety of machine learning problem in predicting target attribute. I participated in both modeling and (big) data pipeline. 
% \\ - Features Store - A centralize system to store features of variety project for reusable. I worked in backend and ETL process of storing/retrieving feature. 
% \\ - Lookalike Audience - Find new audience similar to input audience based on variable traits. I am main contribute, worked in researching/modeling, data pipeline and backend. 
% \\ - AutoML - A machine learning system, integrate with feature store, that automate the process of optimizing (ads) campaigns, designed to operate with minimal human intervention. I contributed in backend, data pipeline, model deployment.

\textbf{Spam Filter} \textit{(2017-2018)}

\begin{itemize}
    \item A spam filter for public comment on news, official page.
    \item \textit{Techstack}: Python NLP - scikit-learn and gensim
    \item \textit{Impact}: The model performance was satisfy requirement and eliminate the need of manual moderation. However, retraining with new manually data weekly is required to keep up with change in spam pattern.
\end{itemize}




\textbf{Chatbot for Worldcup} \textit{(2017-2018)}

\begin{itemize}
    \item An information retrieval chatbot that help end-user (Zalo user) get useful information about World Cup 2018.
    \item This chatbot is not intelligent, and has none of chatgpt inside. The chatbot is built using multiple simple machine learning model, combine with clever-crafted logic and rich hand-crafted knowledge base.
    \item \textit{Techstack}: 
    \begin{itemize}
            \item NLP: Sentence classification, NER, and typo correction. Python (gensim, scikit-learn)
            \item Backend : Handling logic - Java servlet
            \item Database: As knowledge base - Neo4j (graph database)
            \item Web Crawler: get real time information such as goal, match result, bet ratio on 188bet - Python (scrapy)
    \end{itemize}
    \item \textit{Impact}: The chatbot was well received by users as well as leadership, and was heavily promoted by media/event team, rack in hundreds of interactions in total. The most used feature is retrieve current bet ratio from 188bet.
\end{itemize}


\textbf{Feature Store} \textit{(2019-2023)}
\begin{itemize}
    \item A centralized system to store features of variety project for reusable.
    \item \textit{Techstack}:
    \begin{itemize}
        \item Backend: Handling logic - Java Servlet
        \item Data Pipeline and Big Data: data extraction, transformation and ingestion. I use Python - spotify/luigi as scheduler. Scala Spark to do data transformation. Data is stored in Hadoop with Kerberos authentication.
        \item Database: Use as operational database for the system and store metadata.
        \item Devops: git, docker, gitlab CI
    \end{itemize} 
    \item \textit{Impact}: The project allow data scientist work in difference project to reuse works of each others, save time on feature engineering similar features. Feature Store is integrate into some of my later projects.
\end{itemize}


\textbf{Pregnant Prediction } \textit{(2021-2022)}
\begin{itemize}
    \item Predict whether a target is pregnant. The researching and modeling process were done with jupyter notebook. Then, I built big data pipeline with Spark to run in production.
    \item \textit{Techstack}:
    \begin{itemize}
        \item Modeling: Python - lightgbm, numpy, pandas, matplotlib, pyspark
        \item Data Pipeline and Big Data: Run prediction on all Zalo user, on monthly basis, then load into feature store. Python - luigi scheduler, Scala (Spark), Hadoop, Kerberos
        \item Devops: git, docker gitlab CI
    \end{itemize} 
    \item \textit{Impact}: Result is use as feature for Milk Demand ads targeting, increase CR by 10-15%.
\end{itemize}

\textbf{Lookalike} \textit{(2021-2022)}
\begin{itemize}
    \item Build lookalike audience with custom filter conditions from given audience. The project help user expands their exist audiences.
    \item \textit{Techstack}:
    \begin{itemize}
        \item Modeling: Python - lightgbm, numpy, pandas, matplotlib, pyspark
        \item Data Pipeline and Big Data: Retrain model monthly,  model inference with all Zalo user. Python - luigi scheduler, Scala (Spark), Hadoop, Kerberos
        \item Backend: receive inference requests from multiple platform. Java - Spring Boot
        \item Database: use as operational database, to store user requests and its status. Mongodb.
        \item Devops: git, docker gitlab CI
    \end{itemize} 
    \item \textit{Impact}: About 30-40 lookalike audiences generated monthly.
\end{itemize}


\textbf{AutoML} \textit{(2022-2024)}
\begin{itemize}
    \item A machine learning system, integrate with feature store, that automate the process of optimizing (ads) campaigns, designed to operate with minimal human intervention.
    \item \textit{Techstack}:
    \begin{itemize}
        \item Modeling: Python - lightgbm, autogluon, numpy, pandas, matplotlib, pyspark
        \item Visualize and reporting: Python - pandas, numpy, matplotlib, plotly, papermill, jupyter notebook
        \item Data Pipeline and Big Data: Retrain model monthly,  model inference with all Zalo user. Python - luigi scheduling, Scala (Spark), Hadoop, Kerberos
        \item Backend: receive inference requests from multiple platform. Java Servlet
        \item Database: use as operational database, Mongodb, MySql
        \item Devops: git, docker, gitlab CI
    \end{itemize} 
    \item \textit{Impact}: CR, CTR of ads campaigns increase from 25\% to 50\%. 
    \item Awarded The Outstanding Project in 2023.
\end{itemize}


\begin{rSection}{Education}
%--copy and paste this region  if you need more--
{\bf VNUHCM - University of Science} \hfill {\em 2013 - 2017 } 
\\ Bachelor of Computer Science % \hfill { GPA: your gpa }
%--copy and paste this region  if you need more--

%--copy and paste this region  if you need more--
{\bf VNUHCM - University of Science} \hfill {\em 2019 - 2021 } 
\\ Master of Science in Information System % \hfill { GPA: your gpa }

%--copy and paste this region  if you need more--

\end{rSection}


%--copy and paste this region  if you need more--
\end{rSection}
%--------------------------------------------------------------------------------
%    PROJECTS
%-----------------------------------------------------------------------------------------------
\begin{rSection}{Research Paper}
%--copy and paste this region  if you need more--
% {\bf Title of the project}{, institution associated with it} \hfill {\em start date- end date}\\
% description of the project\\\\
%--copy and paste this region  if you need more--

2017 - Combining Convolution and Recursive Neural Networks for Sentiment Analysis (\href{https://dl.acm.org/doi/abs/10.1145/3155133.3155158}{link}  )
(\href{https://github.com/ttpro1995/soICT2017/releases/tag/v1.1}{full-text})

2020 - Lightweight solution to background noise in crowd counting 
(\href{https://ieeexplore.ieee.org/document/9335834}{link}  )
(\href{https://github.com/ttpro1995/NICS2020_paper/releases/tag/v1}{full-text})

\end{rSection}
%--------------------------------------------------------------------------------
%    ACTIVITIES
%-----------------------------------------------------------------------------------------------
% \begin{rSection}{Activities}
% %--copy and paste this region  if you need more--
% {\bf Title of the activities}{, institution associated with it} \hfill {\em start date- end date}\\
% description of the activities\\\\
% %--copy and paste this region  if you need more--
% \end{rSection}
%----------------------------------------------------------------------------------------
%	SKILLS SECTION
%----------------------------------------------------------------------------------------
% \begin{rSection}{Skills}
% {\bf Programming Languages and Frameworks }
% \\ - Comfortable with Java, Python, shell script
% \\ - Data Science: Data transformation and modeling with Numpy, Pandas, Scikit-learn, lightgbm ,Autogluon
% \\ - Big Data: Experienced in building data pipeline with Apache Hadoop, Apache Spark and Luigi (Spotify).
% \\ - Deep learning: Worked with Pytorch. I did some research on NLP (Bachelor degree research) and Computer Vision (master degree research). Read my research papers for more details.
% \\ - Backend: Build microservice with SpringBoot, Java Servlet, Python flask, Python fastapi, RPC (Apache Thrift, gRPC), REST API. 
% \\ - Database: Worked with MySQL, MongoDB, Neo4j. 
% \\ - Devops: Worked with git, docker/docker-compose, and Gitlab CI. 
% \end{rSection}

% \newpage



\begin{rSection}{Certificate}
{\bf Coursera certificate }

Specialization Certificate 
\\ - Mathematics for Machine Learning and Data Science (\href{https://coursera.org/share/ae3f40effbceedf3488075d566a95551}{link})
\\ - Machine Learning Engineering for Production (MLOps) (\href{https://coursera.org/share/ce5ec378a0075b8ffdbd5b235f8a84bd}{link})


{\bf Languages}
\\Vietnamese: Native
\\English: C2 proficient (\href{https://www.efset.org/cert/o5HAWP}{Certificate})
\end{rSection}



% \begin{rSection}{Awards and Scholarships} 
% {\bf Name of awrd/scholarship}{ , institution associated with the award
% } \hfill{\em date awarded}
% \\description of the award/scholarship\\

% \end{rSection}

\end{document}----------------------------

