%%%%%%%%%%%%%%%%%%%%%%%%%%%%%%%%%%%%%%%%%
% Important note:
% This template requires the resume.cls file to be in the same directory as the
% .tex file. The resume.cls file provides the resume style used for structuring the
% document.
%
%%%%%%%%%%%%%%%%%%%%%%%%%%%%%%%%%%%%%%%%%

%----------------------------------------------------------------------------------------
%	PACKAGES AND OTHER DOCUMENT CONFIGURATIONS
%----------------------------------------------------------------------------------------

\documentclass{resume} % Use the custom resume.cls style
\usepackage{hyperref}
\usepackage[left=0.75in,top=0.6in,right=0.75in,bottom=0.6in]{geometry} % Document margins
\newcommand{\tab}[1]{\hspace{.2667\textwidth}\rlap{#1}}
\newcommand{\itab}[1]{\hspace{0em}\rlap{#1}}
\name{Thai Thien} % Your name
\address{Ho Chi Minh city, Viet Nam} % Your address
%\address{123 Pleasant Lane \\ City, State 12345} % Your secondary addess (optional)
\address{(+84) 082 449 7847\\ thien.thai.work@gmail.com} % Your phone number and email

\begin{document}

%----------------------------------------------------------------------------------------
%	EDUCATION SECTION
%----------------------------------------------------------------------------------------

% \begin{rSection}{Education}
% %--copy and paste this region  if you need more--
% {\bf VNUHCM - University of Science} \hfill {\em 2013 - 2017 } 
% \\ Bachelor of Computer Science % \hfill { GPA: your gpa }
% %--copy and paste this region  if you need more--

% %--copy and paste this region  if you need more--
% {\bf VNUHCM - University of Science} \hfill {\em 2019 - 2021 } 
% \\ Master of Science in Information System % \hfill { GPA: your gpa }

% %--copy and paste this region  if you need more--

% \end{rSection}


\begin{rSection}{Skills}
{\bf Programming Languages and Frameworks }
\\ - Languages: Java, Python,  Bash shell (6+ years)
\\ - Data Science: Data transformation and modeling with Numpy, Pandas, Scikit-learn, lightgbm ,Autogluon
\\ - Big Data: Experienced in building data pipeline with Apache Hadoop, Apache Spark. Build data workflow with Luigi (Spotify), Airflow.
\\ - Visualize and report: Jupyter Notebook, matplotlib and plotly, papermill
\\ - Deep learning: Worked with Pytorch, Tensorflow. 
\\ - Focus on NLP (Bachelor) and Computer Vision (Master). 
\\ - Backend: Setup microservice with SpringBoot, Java Servlet, Python Flask, Python fastapi, RPC (Apache Thrift, gRPC), REST API. 
\\ - Database: MySQL, PostgreSQL, MongoDB, Neo4j. 
\\ - Devops: Git, Gitlab CI, Docker.
\end{rSection}



%----------------------------------------------------------------------------------------
%	EXPERIENCE SECTION
%----------------------------------------------------------------------------------------
\begin{rSection}{Work Experience}
%--copy and paste this region  if you need more--

{\bf \underline{VNG Corporation}} \hfill {\em 2017 - now}
\\Senior Data Scientist
% \\ - Work on variety of machine learning problem in predicting target attribute. I participated in both modeling and (big) data pipeline. 
% \\ - Features Store - A centralize system to store features of variety project for reusable. I worked in backend and ETL process of storing/retrieving feature. 
% \\ - Lookalike Audience - Find new audience similar to input audience based on variable traits. I am main contribute, worked in researching/modeling, data pipeline and backend. 
% \\ - AutoML - A machine learning system, integrate with feature store, that automate the process of optimizing (ads) campaigns, designed to operate with minimal human intervention. I contributed in backend, data pipeline, model deployment.

\textbf{Spam Filter} \hfill {\em \textit{2017 - 2018}}

\begin{itemize}
    \item A spam filter for public comment on news, official page.
    \item \textit{Techstack}: Python NLP - scikit-learn and gensim
    \item \textit{Impact}: Eliminate the need of manual moderation, reducing cost of operation
\end{itemize}




\textbf{Chatbot for Worldcup} \hfill {\em \textit{2017 - 2018}}

\begin{itemize}
    \item A chatbot that help Zalo users get useful information during the height of World Cup 2018. Increasing user interaction with Zalo platform without relying on operation team.
    \item Built based on multiple Machine Learning model, utilizing hand-crafted knowledge and rule-based logic.
    \item \textit{Techstack}: 
    \begin{itemize}
            \item NLP: Sentence classification, NER, and typo correction. Python (gensim, scikit-learn)
            \item Backend : Java servlet
            \item Database: Neo4j (graph database)
            \item Web Crawler: real time information acquisition such as goal, match result, bet ratio - Python scrappy.
    \end{itemize}
    \item \textit{Impact}: The chatbot was well received by users as well as leadership team, resulting it and got heavily promoted by media/event team to public. Traffic to Zalo platform got increased by 50\%.
\end{itemize}

\newpage
\textbf{Feature Store} \hfill {\em \textit{2019 - 2023}}
\begin{itemize}
    \item An initiation to centralize features across different teams for storing, processing, and accessing commonly used features, promoting feature reusability, feature discovery.
    \item \textit{Techstack}:
    \begin{itemize}
        \item Backend: Java Servlet
        \item Data Pipeline and Big Data: Perform data extraction, transformation and ingestion.
        \begin{itemize}
            \item Scheduler: Python + Luigi 
            \item Data transformation: Scala Spark 
            \item Data Storage: Hadoop with Kerberos authentication.
        \end{itemize}
        \item MySQL as operational database and feature metadata.
        \item Git, Docker, Gitlab CI
    \end{itemize} 
    \item \textit{Impact}: Drastically reduce time spent on feature engineering. Feature Store has become standard integration for Data Insight team. 
\end{itemize}


\textbf{Pregnancy Prediction } \hfill {\em \textit{2021 - 2022}}
\begin{itemize}
    \item Predict whether a woman is pregnant. Served as analysis feature for downstream ads targeting campaign.
    \item \textit{Techstack}:
    \begin{itemize}
        \item Modeling: Python - lightgbm, numpy, pandas, matplotlib, pyspark
        \item Data Pipeline and Big Data: Run prediction on all Zalo user, then load into feature store. Python + luigi as scheduler, Scala (Spark), Hadoop for big data processing.
        \item Devops: git, docker gitlab CI
    \end{itemize} 
    \item \textit{Impact}: Use as feature for Milk Demand ads targeting, increase CR by 10-15\%.
\end{itemize}

\textbf{Lookalike} \hfill {\em \textit{2021 - 2022}}
\begin{itemize}
    \item Build lookalike audience with custom filter conditions from given audience. The project help user expands their exist audiences.
    \item \textit{Techstack}:
    \begin{itemize}
        \item Modeling: Python - lightgbm, numpy, pandas, matplotlib, pyspark
        \item Data Pipeline and Big Data: Retrain model monthly,  model inference with all Zalo user. Python - luigi scheduler, Scala (Spark), Hadoop, Kerberos
        \item Backend: receive inference requests from multiple platform. Java - Spring Boot
        \item Database: use as operational database, to store user requests and its status. Mongodb.
        \item Devops: git, docker gitlab CI
    \end{itemize} 
    \item \textit{Impact}: About 30-40 lookalike audiences generated monthly.
\end{itemize}


\textbf{AutoML} \hfill {\em \textit{2022 - 2024}}
\begin{itemize}
    \item A machine learning system, integrate with feature store, that automate the process of optimizing (ads) campaigns, designed to operate with minimal human intervention.
    \item \textit{Techstack}:
    \begin{itemize}
        \item Modeling: Python - lightgbm, autogluon, numpy, pandas, matplotlib, pyspark
        \item Visualize and reporting: Python - pandas, numpy, matplotlib, plotly, papermill, jupyter notebook
        \item Data Pipeline and Big Data: Retrain model monthly,  model inference with all Zalo user. Python - luigi scheduling, Scala (Spark), Hadoop, Kerberos
        \item Backend: receive inference requests from multiple platform. Java Servlet
        \item Database: use as operational database, Mongodb, MySql
        \item Devops: git, docker, gitlab CI
    \end{itemize} 
    \item \textit{Impact}: CR, CTR of ads campaigns increase from 25\% to 50\%. 
    \item Awarded The Outstanding Project in 2023.
\end{itemize}

{\bf \underline{Backend Developer at ShareCarForAds.com}} \hfill {\em 2016 - 2017}\\
- I worked as backend developer. I built backend service to help with data ingestion from IoT devices. 

{\bf \underline{Android Developer at Piksal JSC}} \hfill {\em 2015 - 2016}\\
- Develop variable applications and games on Android. Work with native Java Android, React Native, and Unity Game Engine. 

\end{rSection}

\begin{rSection}{Education}
%--copy and paste this region  if you need more--
%--copy and paste this region  if you need more--
{\bf VNUHCM - University of Science} \hfill {\em 2019 - 2021 } 
\\ Master of Science in Information System  %\hfill { GPA: 3.8/4.0 }

{\bf VNUHCM - University of Science} \hfill {\em 2013 - 2017 } 
\\ Bachelor of Computer Science  %\hfill { GPA: 3.8/4.0 }
%--copy and paste this region  if you need more--
\end{rSection}
%--------------------------------------------------------------------------------
%    PROJECTS
%-----------------------------------------------------------------------------------------------
\begin{rSection}{Research Paper}
%--copy and paste this region  if you need more--
% {\bf Title of the project}{, institution associated with it} \hfill {\em start date- end date}\\
% description of the project\\\\
%--copy and paste this region  if you need more--
2020 - Lightweight solution to background noise in crowd counting 
(\href{https://ieeexplore.ieee.org/document/9335834}{link}  )
(\href{https://github.com/ttpro1995/NICS2020_paper/releases/tag/v1}{full-text})

2017 - Combining Convolution and Recursive Neural Networks for Sentiment Analysis (\href{https://dl.acm.org/doi/abs/10.1145/3155133.3155158}{link}  )
(\href{https://github.com/ttpro1995/soICT2017/releases/tag/v1.1}{full-text})
\end{rSection}
%--------------------------------------------------------------------------------
%    ACTIVITIES
%-----------------------------------------------------------------------------------------------
% \begin{rSection}{Activities}
% %--copy and paste this region  if you need more--
% {\bf Title of the activities}{, institution associated with it} \hfill {\em start date- end date}\\
% description of the activities\\\\
% %--copy and paste this region  if you need more--
% \end{rSection}
%----------------------------------------------------------------------------------------
%	SKILLS SECTION
%----------------------------------------------------------------------------------------
% \begin{rSection}{Skills}
% {\bf Programming Languages and Frameworks }
% \\ - Comfortable with Java, Python, shell script
% \\ - Data Science: Data transformation and modeling with Numpy, Pandas, Scikit-learn, lightgbm ,Autogluon
% \\ - Big Data: Experienced in building data pipeline with Apache Hadoop, Apache Spark and Luigi (Spotify).
% \\ - Deep learning: Worked with Pytorch. I did some research on NLP (Bachelor degree research) and Computer Vision (master degree research). Read my research papers for more details.
% \\ - Backend: Build microservice with SpringBoot, Java Servlet, Python flask, Python fastapi, RPC (Apache Thrift, gRPC), REST API. 
% \\ - Database: Worked with MySQL, MongoDB, Neo4j. 
% \\ - Devops: Worked with git, docker/docker-compose, and Gitlab CI. 
% \end{rSection}

% \newpage



\begin{rSection}{Certificate}
{\bf Coursera certificate }

Specialization Certificate 
\\ - Mathematics for Machine Learning and Data Science (\href{https://coursera.org/share/ae3f40effbceedf3488075d566a95551}{link})
\\ - Machine Learning Engineering for Production (MLOps) (\href{https://coursera.org/share/ce5ec378a0075b8ffdbd5b235f8a84bd}{link})


{\bf Languages}
\\Vietnamese: Native
\\English: C2 proficient (\href{https://www.efset.org/cert/o5HAWP}{Certificate})
\end{rSection}



% \begin{rSection}{Awards and Scholarships} 
% {\bf Name of awrd/scholarship}{ , institution associated with the award
% } \hfill{\em date awarded}
% \\description of the award/scholarship\\

% \end{rSection}

\end{document}----------------------------

